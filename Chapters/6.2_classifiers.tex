\section{Classifiers Implementation}

Following the selection of the dataset, the \gls{DNA} sequence records were randomly rearranged and assigned to the training and test/validation sets. A suitable data preprocessing strategy was selected in order to turn the \gls{DNA} sequences, into a numerical representation. This format was necessary to comply with the input requirements of the classification model, which takes only numerical data.

At this stage, there will be either a collection of pre-calculated features, the descriptors, or labeled and encoded training data. The data will be used to train the classification model regardless of the previous choice.

Various different classification models were created to serve both approaches of setting up the data, and the following sections will provide an insight for each one of them.

\subsection{MLP}

\subsection{CNN}

\subsection{RNN}

\subsection{LSTM / BiLSTM}

\subsection{CNN-LSTM / CNN-BiLSTM}