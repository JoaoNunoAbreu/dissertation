\chapter{Development and Implementation}\label{cha:dev}

\section{Development Strategy}\label{sec:dev_strat}

The main goal of this study was to create a tool that can automatically classify \gls{DNA} sequences using \gls{ML}/\gls{DL} models, followed by its integration into \textit{ProPythia}. The other key objective was to integrate automated machine learning classifiers into the \textit{OmniumAI} software platforms.

However, it is crucial to understand the steps and the technologies necessary to build such tools. These tools were built with \textit{Python}, which is a high-level interpreted general-purpose programming language that supports vast and extensive external libraries that are constantly evolving. \textit{PyCharm Code Editor} was used to combine \textit{Python} code development, while also improving usability and user experience. Additionally, the \textit{Anaconda} software application was employed to facilitate package management and distribution. Then, for building the entire classification pipeline, the \textit{PyTorch} framework was used as it is one of the most popular free open source and powerful \gls{DL} frameworks.

In an \gls{ML} project, after selecting the dataset, the first step is the data processing task since, in most cases, the data is not ready to be fed into a model. The sequences need first to be converted into a numerical representation, and how this step is executed has a substantial influence on the model's ultimate performance. 
As mentioned in Section~\ref{sec:workflow}, data processing can be subdivided into feature extraction and feature selection. The first is the process of transforming raw data into suitable modelling features, and the second consists of filtering irrelevant features, keeping the essential ones. If working with a \gls{DL} model, the user may not need to perform this step since these kind of models can extract the features from the training data on their own.

In this study, descriptors were chosen to be the features for shallow \gls{ML} classification models, which is a manual feature extraction process. For \gls{DL} models, they only need the sequences as input, but they still need to be in a numerical format. Encoders were utilized to address this problem, since they can create numerical representations of a string of letters.

%%%%%%%%%%%%%%%%%%%%%%%%%%%%%%%%%%%%%%%%%%%%%%%%%%%%%%%%%%%%%%%%%%%%%%%%%%%%%%%%%%%%%%%%%%%%%%%%%%%%%%%%%%%%%%%%%%%%%


\section{Setting up the Data}

\subsection{Descriptors}
\label{cha:descriptors}

In this work, we first examined the state-of-the-art in computationally predicting ... and then comprehensively assessed the predictive power of eight traditional machine learning methods and 17 feature types often used in prior research.

Based on the sequence and physicochemical characteristics, a total of 17 different kinds of features were created and evaluated. Table~\ref{tab:descriptors_classes} provides the list of all the descriptors as well as their respective group.

\begin{table}[ht]
	\caption{List of descriptors calculated.}
	\label{tab:descriptors_classes}
    \centering
    \begin{tabular}{ll}
    	\toprule
    	\textbf{Descriptor groups} & \textbf{Descriptor} \\\midrule
    	
    	\multirow{3}{*}{Psychochemical} & Length \\
    	& GC Content\\
    	& AT Content\\\midrule
    	
    	\multirow{6}{*}{Nucleic Acid Composition} & Nucleic Acid Compostion (NAC) \\
    	
    	& Dinucleotide Acid Compostion (DNC)\\
    	& Trinucleotide Acid Compostion (TNC)\\
    	& Composition of K-spaced nucleic acid pairs (CKSNAP)\\
    	& K-mer\\
    	& Accumulated Nucleotide Frequency (ANF)\\\midrule
    	
    	\multirow{6}{*}{Autocorrelation} & Dinucleotide-based Auto Covariance (DAC) \\
    	& Dinucleotide-based Cross Covariance (DCC)\\
    	& Dinucleotide-based Auto-Cross Covariance (DACC)\\
    	& Trinucleotide-based Auto Covariance (TAC) \\
    	& Trinucleotide-based Cross Covariance (TCC) \\
    	& Trinucleotide-based Auto-Cross Covariance (TACC)\\\midrule
    	
    	\multirow{2}{*}{Pseudo Nucleic Acid Composition} & Pseudo Dinucleotide Compostion (PseDNC) \\
    	& Pseudo K-tupler Compostion (PseKNC) \\
    	
        
    	\bottomrule
    \end{tabular}
\end{table}

Both the psychochemical properties and the nucleic acid composition are the most straight-forward approaches to represent the DNA sequences.

\paragraph{Length}

Length descriptor is a simple descriptor that calculates the length of a sequence.

\paragraph{GC Content}

\gls{GC} content feature enconding represents the quantity of guanine and cytosine nucleotides in a sequence. It can be calculated as follows:

\begin{equation}\label{eq:gc_content}
    x = \frac{N_{(C)} + N_{(G)}}{N}
\end{equation}


where $N_{(C)}$ donates the number of the nucleotide C in the sequence, $N_{(G)}$ the number of the nucleotide G and $N$ the length of the sequence.

\paragraph{AT Content}

\gls{AT} content feature enconding represents the quantity of adenine and thymine nucleotides in a sequence. It can be calculated as follows:


\begin{equation}\label{eq:at_content}
    x = \frac{N_{(A)} + N_{(T)}}{N}
\end{equation}

where $N_{(A)}$ donates the number of the nucleotide A in the sequence, $N_{(T)}$ the number of the nucleotide T and $N$ the length of the sequence.


\paragraph{Nucleic Acid Composition}

As one of the commonly used methods to represent \gls{DNA} sequences, the \gls{NAC} encoding [25, 26] reflects the nucleotides frequencies of the sequence. The frequencies of all four natural nucleotides ('A', 'C', 'G' and 'T') can be calculated as:

\begin{equation}\label{eq:NAC}
    f(i) = \frac{N_{(i)}}{N}, i \in \left\{A,C,T,G\right\}
\end{equation}

where $N_{(i)}$ donates the number of nucleotide type and $N$ represents the length of a \gls{DNA} sequence.

\paragraph{Di-Nucleotide Composition}

\gls{DNC} feature encoding [27, 28] represents the composition of continuous dinucleotide pairs in a \gls{DNA} sequence. There are 16 descriptors in \gls{DNC} feature encoding, which can be defined as:

\begin{equation}\label{eq:DNC}
    D(i,j) = \frac{N_{(ij)}}{N-1}, i,j \in \left\{A,C,T,G\right\}
\end{equation}


where $N_{(ij)}$ donates the number of dinucleotide represented by nucleotide types $i$ and $j$.

\paragraph{Tri-Nucleotide Composition}

\gls{TNC} feature encoding [29, 30] represents the composition of the composition of continuous trinucleotide pairs in a \gls{DNA} sequence. There are 64 descriptors in \gls{TNC} feature encoding ('AAA', 'AAC', 'AAG', 'AAT', ..., 'TTT'), which can be defined as:

\begin{equation}\label{eq:TNC}
    D(i,j,k) = \frac{N_{(ijk)}}{N-2}, i,j,k \in \left\{A,C,T,G\right\}
\end{equation}

where $N_{(ijk)}$ donates the number of trinucleotides represented by nucleotide types $i$, $j$ and $k$.


\paragraph{Composition of K-spaced nucleic acid pairs}

\gls{CKSNAP} feature encoding [21, 22] represents the composition of nucleotide pairs that are K-steps away from each other in a segment. Specifically, we calculated the frequency of a nucleotide pair with the two nucleotides at positions $i$ and $i + K + 1$, respectively, where $i$ = 1, ..., ($l$ − $K$ − 1) and l being the length of the sequence. For example, given the sequence $ACGTACGT$ and $K$ = 2, the nucleotide AT will occur twice in the sequence, where A and T occur at positions 1 and 4 and also at positions 5 and 8. 

It is important to note that there are only a total of 16 possible nucleotide pairs regardless of the value of $K$. This coding system reflects the short-range interactions of nucleic acids within a \gls{DNA} sequence segment.

\paragraph{K-mer}

The K-mer encoding [34, 35] calculates the occurrence frequencies of k neighboring nucleotide in a \gls{DNA} sequence, which was commonly used in the field of enhancer identification and regulatory sequence prediction (2). The K-mer (k = 4) descriptor can be defined as:

\begin{equation}\label{eq:K-mer}
    K(i) = \frac{N_{(i)}}{N}, i \in \left\{AAAA, AAAC, AAAG,...,TTTT\right\}
\end{equation}

where $N_{i}$ donates the number of types $i$ descriptor of K-mer and $N$ representes the length of the DNA sequence.

The implemented K-mer descriptor also includes the \gls{RCKmer}. The \gls{RCKmer} encoding [36] is a variant of K-mer descriptor, which calculates the occurrence frequencies of reverse compliment k neighboring nucleotide in the \gls{DNA} sequence. For example, there are 16 types of 2-mers in a \gls{DNA} sequence. Among them, 'TT' is reverse compliment with 'AA'. Thus, there are only 10 types of 2-mers in the \gls{RCKmer} approach (i.e. 'AA', 'AC', 'AG', 'AT', 'CA', 'CC', 'CG', 'GA', 'GC' and 'TA') by removing the reverse complimentary K-mers.

\paragraph{Accumulated Nucleotide Frequency}

The nucleotide density and distribution of each nucleotide in a \gls{DNA} segment are represented by the \gls{ANF} feature encoding scheme [17]. The formula below describes how to calculate the \gls{ANF} of a \gls{DNA} segment of length L.

\begin{equation}\label{eq:ANF}
    d_{l} = \frac{1}{l}\sum_{j=1}^{l}f(n_{j}), f(n_{j}) = \begin{cases}1 & n_{j} = q\\0 & other\end{cases}, l = 1,...,L
\end{equation}

where $n_{j}$ represents the nucleotide at the j-th position and $q \in (A,C,T,G)$. Taking the sequence of 'TGCTACGC' as an example, when $l$ = 3, the nucleotide at the $l$-th position is C and the density of this position is calculated as $d_{3} = \frac{1}{3}\sum_{j=1}^{3}f(n_{j}) = \frac{1}{3} [f(C) + f(A) + f(C)] = \frac{1}{3} [1+0+1] = 0.667$. The density of all L positions can be similarly calculated. However, if we calculate the ANF for each position, we would get L values, which is not a fixed length vector. To avoid this, we only calculated the ANF for 3 positions, which were at 25, 50 and 75\% of the length of the sequence.

\paragraph{Dinucleotide-based Auto Covariance}

The first autocorrelation descriptor is called \gls{DAC}. Autocorrelation, as one of the multivariate modeling tools, can transform the DNA sequences of different lengths into fixed-length vectors by measuring the correlation between any two physicochemical properties. Autocorrelation results in two kinds of variables: autocorrelation (AC) between the same property, and cross-covariance (CC) between two different properties.

It has been proved that DNA physicochemical properties play important roles in gene expression regulation [31-33]. For example, DNA physicochemical property is evolutionarily more constrained than the underlying actual sequence, and the topography-informed constrained regions usually correlate with functional noncoding elements such as enhancers [34].

Listed in Table~\ref{tab:38_di}, Table~\ref{tab:12_tri} are 38 dinucleotide physicochemical properties and 12 trinucleotide physicochemical properties that can be used to generate various different modes of dinucleotide and trinucleotide, respectively, autocorrelation features. Because so far not much physicochemical property data are available for the K-tuple nucleotides when $K = 4$ (tetranucleotides) and above, the current study was limited to dinucleotide and trinucleotide autocorrelation descriptors only. Nevertheless, the formulations presented here can be easily extended to cover the case of $K \geq 4$ when the corresponding physicochemical property data are available.


\begin{footnotesize}
    \begin{longtable}{ccc}
        \caption{List of 38 physicochemical properties of dinucleotides in DNA.~\cite{Chen2014PseKNC:Composition}}
        \label{tab:38_di}
        \endfirsthead
        \endhead
        \toprule
        \textbf{Number} & \textbf{Description} & \textbf{Reference}\\\midrule
        
        1 & Base stacking	& [37] \\\midrule
        2 & Protein-induced deformability	& [38] \\\midrule
        3 & B-DNA twist	& [39] \\\midrule
        4 & Dinucleotide GC content	& [40] \\\midrule
        5 & A-philicity	& [41] \\\midrule
        6 & Propeller twist	& [42] \\\midrule
        7 & Duplex stability (free energy)	& [43] \\\midrule
        8 & Duplex stability (disrupt energy)	& [44] \\\midrule
        9 & DNA denaturation	& [45] \\\midrule
        10 & Bending stiffness	& [46] \\\midrule
        11 & Protein-DNA twist	& [38] \\\midrule
        12 & Stabilizing energy of Z-DNA	& [47] \\\midrule
        13 & Aida\_BA\_transition	& [48] \\\midrule
        14 & Breslauer\_dG	& [44] \\\midrule
        15 & Breslauer\_dH	& [44] \\\midrule
        16 & Breslauer\_dS	& [44] \\\midrule
        17 & Electron interaction	& [40] \\\midrule
        18 & Hartman\_trans\_free\_energy	& [49] \\\midrule
        19 & Helix-coil\_transition	& [50] \\\midrule
        20 & Ivanov\_BA\_transition	& [41] \\\midrule
        21 & Lisser\_BZ\_transition	& [51] \\\midrule
        22 & Polar\_interaction	& [52] \\\midrule
        23 & SantaLucia\_dG	& [53] \\\midrule
        24 & SantaLucia\_dH	& [53] \\\midrule
        25 & SantaLucia\_dS	& [53] \\\midrule
        26 & Sarai\_flexibility	& [54] \\\midrule
        27 & Stability	& [55] \\\midrule
        28 & Stacking\_energy	& [37] \\\midrule
        29 & Sugimoto\_dG	& [53] \\\midrule
        30 & Sugimoto\_dH	& [53] \\\midrule
        31 & Sugimoto\_dS	& [53] \\\midrule
        32 & Watson-Crick\_interaction	& [56] \\\midrule
        33 & Twist	& [57] \\\midrule
        34 & Tilt	& [57] \\\midrule
        35 & Roll	& [57] \\\midrule
        36 & Shift	& [57] \\\midrule
        37 & Slide	& [57] \\\midrule
        38 & Rise	& [57] \\
        
        \bottomrule
        
    \end{longtable}
\end{footnotesize}
    
\begin{footnotesize}
    \begin{longtable}{ccc}
        \caption{List of 12 physicochemical properties of trinucleotides in DNA.~\cite{Chen2014PseKNC:Composition}}
        \label{tab:12_tri}
        \endfirsthead
        \endhead
        \toprule
        \textbf{Number} & \textbf{Description} & \textbf{Reference}\\\midrule
        
        1	& Bendability (DNase)	& [58]\\\midrule
        2	& Bendability (consensus)	& [58]\\\midrule
        3	& Trinucleotide GC content	& [40]\\\midrule
        4	& Nucleosome positioning	& [59]\\\midrule
        5	& Consensus-roll	& [40], [52]\\\midrule
        6	& Consensus-rigid	& [40], [52]\\\midrule
        7	& DNase I	& [31]\\\midrule
        8	& DNase I-rigid	& [31]\\\midrule
        9	& MW-daltons	& [40]\\\midrule
        10	& MW-kg &	[40]\\\midrule
        11	& Nucleosome	& [60]\\\midrule
        12	& Nucleosome-rigid	& [60]\\
        
        \bottomrule
        
    \end{longtable}
\end{footnotesize}

Suppose a DNA sequence D with L nucleic acid residues; i.e. 

\begin{equation}\label{eq:d-seq}
    D = R_{1}\;R_{2}\;R_{3}\;...\;R_{L}
\end{equation}

denotes the nucleic acid residue at the sequence position $i \in [1,2,...,L]$ and $R_{i} \in \lbrace A,C,T,G\rbrace$.

The \gls{DAC} measures the correlation of the same physicochemical index between two dinucleotide separated by a distance of lag along the sequence, which can be calculated as:

\begin{equation}\label{eq:DAC}
    DAC(u,lag) = \frac{\sum_{i=1}^{L-lag-1}(P_{u}(R_{i}\;R_{i+1}) - \overline{P_{u}})(P_{u}(R_{i+lag}\;R_{i+lag+1}) - \overline{P_{u}})}{L-lag-1}
\end{equation}

where $u$ is a physicochemical index, $L$ is the length of the DNA sequence, $P_{u}(R_{i}R_{i+1})$ means the numerical value of the physicochemical index $u$ for the dinucleotide $R_{i}R_{i+1}$ at position $i$, $\overline{P_{u}}$ is the average value for physicochemical index $u$ along the whole sequence:

\begin{equation}\label{eq:DAC-PU}
    \overline{P_{u}} = \frac{\sum_{j=1}^{L-1}P_{u}(R_{j}R_{j+1})}{L-1}
\end{equation}

In such a way, the length of \gls{DAC} feature vector is $N*LAG$, where $N$ is the number of physicochemical indices and $LAG$ is the maximum of $lag, lag \in [1,2,...,LAG]$.

\paragraph{Dinucleotide-based Cross Covariance}
Given a DNA sequence D (Eq.~\ref{eq:d-seq}), the \gls{DCC} approach measures the correlation of two different physicochemical indices between two dinucleotides separated by lag nucleic acids along the sequence, which can be calculated by:

\begin{equation}\label{eq:DCC}
    DCC(u_{1},u_{2},lag) = \frac{\sum_{i=1}^{L-lag-1}(P_{u_{1}}(R_{i}\;R_{i+1}) - \overline{P_{u_{1}}})(P_{u_{2}}(R_{i+lag}\;R_{i+lag+1}) - \overline{P_{u_{2}}})}{L-lag-1}
\end{equation}

where $u_{1}$, $u_{2}$ are two different physicochemical indices, $L$ is the length of the DNA sequence, $P_{u_{1}}(R_{i}R_{i+1}) (P_{u_{2}}(R_{i}R_{i+1}))$ is the numerical value of the physicochemical index $u_{1}(u_{2})$ for the dinucleotide $R_{i}R_{i+1}$ at position $i$, $\overline{P_{u_{1}}}\;(\overline{P_{u_{2}}})$ is the average value for physicochemical index value $u_{1}$, $u_{2}$ along the whole sequence:

\begin{equation}\label{eq:DAC-PU2}
    \overline{P_{u}} = \frac{\sum_{j=1}^{L-1}P_{u}(R_{j}R_{j+1})}{L-1}
\end{equation}

In such a way, the length of \gls{DCC} feature vector is $N*(N-1)*LAG$, where $N$ is the number of physicochemical indices and $LAG$ is the maximum of $lag, lag \in [1,2,...,LAG]$.

\paragraph{Dinucleotide-based Auto-Cross Covariance}
\gls{DACC} is a combination of \gls{DAC} and \gls{DCC}. Therefore, the length of the \gls{DACC} feature vector is $N*N*LAG$, where $N$ is the number of physicochemical indices and $LAG$ is the maximum of $lag, lag \in [1,2,...,LAG]$.


\paragraph{Trinucleotide-based Auto Covariance}
Given a DNA sequence D (Eq.~\ref{eq:d-seq}), the \gls{TCC} approach measures the correlation of the same physicochemical index between two trinucleotides separated by \textit{lag} nucleic acids along the sequence, which can be calculated as:

\begin{equation}\label{eq:tac}
    TAC(lag,u) = 
\frac
{
\sum_{i=1}^{L-lag-2}(P_{u}(R_{i}R_{i+1}R_{i+2}) - \overline{P_{u}})(P_{u}(R_{i+lag}R_{i+lag+1}R_{i+lag+2}) - \overline{P_{u}})
}
{
L-lag-2
}
\end{equation}

where $u$ is a physicochemical index, $L$ is the length of the DNA sequence, $P_{u}(R_{i}R_{i+1}R_{i+2})$ means the numerical value of the physicochemical index $u$ for the trinucleotide $R_{i}R_{i+1}R_{i+2}$ at position $i$, $\overline{P_{u}}$ is the average value for physicochemical index $u$ along the whole sequence:

\begin{equation}\label{eq:tac-2}
    \overline{P_{u}} = \frac{\sum_{j=1}^{L-2}P_{u}(R_{j}R_{j+1}R_{j+2})}{L-2}
\end{equation}

In such a way, the length of \gls{TAC} feature vector is $N*LAG$, where $N$ is the number of physicochemical indices and $LAG$ is the maximum of $lag, lag \in [1,2,...,LAG]$.

\paragraph{Trinucleotide-based Cross Covariance}

Given a DNA sequence D (Eq.~\ref{eq:d-seq}), the \gls{TCC} approach measures the correlation of two different physicochemical indices between two trinucleotides separated by \textit{lag} nucleic acids along the sequence, which can be calculated by:

\begin{equation}\label{eq:TCC}
    TCC(u_{1},u_{2},lag) = \frac{\sum_{i=1}^{L-lag-2}(P_{u_{1}}(R_{i}\;R_{i+1}\;R_{i+2}) - \overline{P_{u_{1}}})(P_{u_{2}}(R_{i+lag}\;R_{i+lag+1}\;R_{i+lag+2}) - \overline{P_{u_{2}}})}{L-lag-2}
\end{equation}

where $u_{1}$, $u_{2}$ are two different physicochemical indices, $L$ is the length of the DNA sequence, $P_{u_{1}}(R_{i}R_{i+1}R_{i+2}) (P_{u_{2}}(R_{i}R_{i+1}R_{i+2}))$ is the numerical value of the physicochemical index $u_{1}(u_{2})$ for the trinucleotide $R_{i}R_{i+1}R_{i+2}$ at position $i$, $\overline{P_{u_{1}}}\;(\overline{P_{u_{2}}})$ is the average value for physicochemical index value $u_{1}$, $u_{2}$ along the whole sequence:

\begin{equation}\label{eq:TCC-PU2}
    \overline{P_{u}} = \frac{\sum_{j=1}^{L-2}P_{u}(R_{j}R_{j+1}R_{j+2})}{L-2}
\end{equation}

In such a way, the length of \gls{TCC} feature vector is $N*(N-1)*LAG$, where $N$ is the number of physicochemical indices and $LAG$ is the maximum of $lag, lag \in [1,2,...,LAG]$.


\paragraph{Trinucleotide-based Auto-Cross Covariance}
\gls{TACC} is a combination of \gls{TAC} and \gls{TCC}. Therefore, the length of the \gls{TACC} feature vector is $N*N*LAG$, where $N$ is the number of physicochemical indices and $LAG$ is the maximum of $lag, lag \in [1,2,...,LAG]$.

\paragraph{Pseudo Dinucleotide Composition}

% https://academic.oup.com/nar/article/41/6/e68/2902382
% https://www.ncbi.nlm.nih.gov/pmc/articles/PMC8138820/

All these methods above were merely based on the nucleic acid composition alone without taking into account the sequence order effect.
\gls{PseNAC} is a kind of powerful approaches to represent the DNA sequences considering both DNA local sequence-order information and long range or global sequence-order effects.

\gls{PseDNC} is an approach incorporating the contiguous local sequence-order information and the global sequence-order information into the feature vector of the DNA sequence. 

Given a DNA sequence D (Eq.~\ref{eq:d-seq}), if the feature vector of D is formulated by its \gls{NAC}, we have:

\begin{equation}\label{eq:PseDNC-0}
    D = [f(A)\;f(C)\;f(G)\;f(T)]^{T}
\end{equation}

where $f(A)$, $f(C)$, $f(G)$ and $f(T)$ are the normalized occurrence frequencies of \gls{A}, \gls{T}, \gls{C} and \gls{G} respectively, in the \gls{DNA} sequence; the symbol $T$ is the transpose operator. As we can see from Eq. \ref{eq:PseDNC-0}, all the sequence-order information is missed if using \gls{NAC} to represent a \gls{DNA} sequence. If using the \gls{DNC} to represent the \gls{DNA} sequence, instead of the four components as shown in Eq. \ref{eq:PseDNC-0}, the corresponding feature vector will contain $4 * 4 = 16$ components, as given below

\begin{equation}\label{eq:PseDNC-1-1}
    D = [f(AA)\;f(AC)\;...\;f(TT)]^{T} = [f_{1}\;f_{2} \;...\;f_{16}]^{T}
\end{equation}

where $f_{1} = f(AA)$ is the normalized occurrence frequency of AA in the \gls{DNA} sequence; $f_{2} = f(AC)$⁠, and so forth. Although the most contiguous local sequence-order information is included in Eq.~\ref{eq:PseDNC-1-1}, none of the global sequence-order information is reflected by the formulation. 

To incorporate the global sequence-order information into the feature vector for the \gls{DNA} sequence, let us consider the following approach. As shown in Eq.~\ref{eq:d-seq}, the first dinucleotide in the \gls{DNA} sequence is $R_{1}R_{2}$, the second dinucleotide is $R_{2}R_{3}$ and so forth; the last one is $R_{L-1}R_{L}$. Thus, by following the similar procedures as described in (7) to reflect the global sequence-order information of a protein with a set of sequence-order-correlated factors, for the \gls{DNA} sequence of Eq.~\ref{eq:d-seq}, we also have the corresponding factors as defined below:

\begin{equation}\label{eq:PseDNC-thetas}
    \begin{cases}
    \theta_{1} = \frac{1}{L-2} \sum_{i=1}^{L-2}\Theta(R_{i}R_{i+1}, R_{i+1}R_{i+2})
    \\ 
    \theta_{2} = \frac{1}{L-3} \sum_{i=1}^{L-3}\Theta(R_{i}R_{i+1}, R_{i+2}R_{i+3})
    \\ 
    \theta_{3} = \frac{1}{L-4} \sum_{i=1}^{L-4}\Theta(R_{i}R_{i+1}, R_{i+3}R_{i+4}) & (\lambda < L)
    \\
    ...
    \\ 
    \theta_{\lambda} = \frac{1}{L-1-\lambda} \sum_{i=1}^{L-1-\lambda}\Theta(R_{i}R_{i+1}, R_{i+\lambda}R_{i+\lambda+1})
    \end{cases}
\end{equation}

where $\theta_{1}$ is called the first-tier correlation factor that reflects the sequence-order correlation between all the most contiguous dinucleotide along a DNA sequence (Figure~\ref{fig:psednc_correlation}A); $\theta_{2}$, the second-tier correlation factor between all the second most contiguous dinucleotide (Figure~\ref{fig:psednc_correlation}B); $\theta_{3}$, the third-tier correlation factor between all the third most contiguous dinucleotide (Figure~\ref{fig:psednc_correlation}C) and so forth.

\begin{figure}[htbp]
    \centering
    \includegraphics[width=0.65\linewidth]{correlation_di.png}
    \caption{Correlations of dinucleotides along a DNA sequence. Adapted from~\cite{Chen2014PseKNC:Composition}}
    \label{fig:psednc_correlation}
\end{figure}


In Eq.~\ref{eq:PseDNC-thetas}, the parameter $\lambda$ is an integer, representing the highest counted rank (or tier) of the correlation along a DNA sequence, and the correlation function is given by:

\begin{equation}\label{eq:PseDNC-5}
    \Theta(R_{i}R_{i+1}, R_{j}R_{j+1}) = \frac{1}{\mu}\sum_{u=1}^{\mu}[P_{u} (R_{i}R_{i+1}) - P_{u}(R_{j}R_{j+1})]^{2}
\end{equation}

where $\mu$ is the number of local DNA structural properties considered that is equal to 6 in the current study as will be explained later in the text; $P_{u} (R_{i}R_{i+1})$⁠, the numerical value of the \textit{u-th} $(u = 1,2,...,\mu$ DNA local property for the dinucleotide $R_{i}R_{i+1}$ at position $i$ and $P_{u}(R_{j}R_{j+1})$⁠, the corresponding value for the dinucleotide $R_{j}R_{j+1}$ at position $j$.

Multiple lines of evidences have indicated that some local DNA structural properties, i.e. angular parameters (twist, tilt and roll) and translational parameters (shift, slide and rise), have important roles in biological processes, such as protein–DNA interactions, formation of chromosomes and higher-order organization of the genetic material (30–32). Accordingly, these six structural properties might have impact on DNA binding of regulatory proteins, either directly by hampering or favoring complex formation or indirectly through the modulation of the chromatin structures and hence the DNA accessibility (33). Listed in Table
~\ref{tab:original_numerical} are their original numerical values derived from (32) for twist $P_{1}(R_{i}R_{i+1})$⁠, tilt $P_{2}(R_{i}R_{i+1})$⁠, roll $P_{3}(R_{i}R_{i+1})$⁠, shift $P_{4}(R_{i}R_{i+1})$⁠, slide $P_{5}(R_{i}R_{i+1})$⁠, and rise $P_{6}(R_{i}R_{i+1})$⁠, respectively, where $R_{i}R_{i+1}$ represents the 16 possible dinucleotides AA, AC, AG, AT, ..., TT. It was these six DNA local physical structural properties that were to be used as correlation functions to derive the \gls{PseDNC} for the current study. Meanwhile, it is also self-evident why $\mu=6$ in Eq.~\ref{eq:PseDNC-5} for the current case.


\begin{table}[ht]

    \caption{Original numerical values for the six DNA dinucleotide physical structures~\cite{Chen2013IRSpot-PseDNC:Composition}}
    \label{tab:original_numerical}
    
    \centering
    \begin{tabular}{ccccccc}

        \toprule
        \textbf{Dinucleotide} & $P_{1}(R_{i}R_{i+1})$ & $P_{2}(R_{i}R_{i+1})$ & $P_{3}(R_{i}R_{i+1})$ & $P_{4}(R_{i}R_{i+1})$ & $P_{5}(R_{i}R_{i+1})$ & $P_{6}(R_{i}R_{i+1})$\\\midrule

        AA & 0.026  & 0.038 & 0.020 & 1.69 & 2.26 & 7.65 \\\midrule
        AC & 0.036  & 0.038 & 0.023 & 1.32 & 3.03 & 8.93 \\\midrule
        AG & 0.031  & 0.037 & 0.019 & 1.46 & 2.03 & 7.08 \\\midrule
        AT & 0.033  & 0.036 & 0.022 & 1.03 & 3.83 & 9.07 \\\midrule
        CA & 0.016  & 0.025 & 0.017 & 1.07 & 1.78 & 6.38 \\\midrule
        CC & 0.026  & 0.042 & 0.019 & 1.43 & 1.65 & 8.04 \\\midrule
        CG & 0.014  & 0.026 & 0.016 & 1.08 & 2.00 & 6.23 \\\midrule
        CT & 0.031  & 0.037 & 0.019 & 1.46 & 2.03 & 7.08 \\\midrule
        GA & 0.025  & 0.038 & 0.020 & 1.32 & 1.93 & 8.56 \\\midrule
        GC & 0.025  & 0.036 & 0.026 & 1.20 & 2.61 & 9.53 \\\midrule
        GG & 0.026  & 0.042 & 0.019 & 1.43 & 1.65 & 8.04 \\\midrule
        GT & 0.036  & 0.038 & 0.023 & 1.32 & 3.03 & 8.93 \\\midrule
        TA & 0.017  & 0.018 & 0.016 & 0.72 & 1.20 & 6.23 \\\midrule
        TC & 0.025  & 0.038 & 0.020 & 1.32 & 1.93 & 8.56 \\\midrule
        TG & 0.016  & 0.025 & 0.017 & 1.07 & 1.78 & 6.38 \\\midrule
        TT & 0.026  & 0.038 & 0.020 & 1.69 & 2.26 & 7.65 \\

        \bottomrule
    \end{tabular}
\end{table}

Before substituting into Eq.~\ref{eq:PseDNC-5}, the original values as listed in Table~\ref{tab:original_numerical} for $P_{u}(R_{i}R_{i+1} (u = 1,2,...,6)$⁠, they were all subjected to a standard conversion (26), as described by the following equation:

\begin{equation}\label{eq:PseDNC-6}
    P_{u}(R_{i}R_{i+1}) = \frac{P_{u}(R_{i}R_{i+1}) - <P_{u}>}{SD(P_{u})}
\end{equation}

where the symbol $<>$ means taking the average of the quantity therein for 16 different dinucleotides (cf. Eq.~\ref{eq:PseDNC-1-1}), and SD means the corresponding standard deviation. The converted values obtained by Eq.~\ref{eq:PseDNC-6} will have a zero mean value for the 16 different dinucleotides and will remain unchanged if going through the same conversion procedure again. Listed in Table~\ref{tab:normalized_values} are the values of $P_{u}(R_{i}R_{i+1} (u = 1,2,...,6)$ obtained via the standard conversion of Eq.~\ref{eq:PseDNC-6} from those of Table~\ref{tab:original_numerical}.

\begin{table}[ht]

    \caption{The normalized values for the six DNA dinucleotide physical structures
    ~\cite{Chen2013IRSpot-PseDNC:Composition}}
    \label{tab:normalized_values}
    
    \centering
    \begin{tabular}{ccccccc}

        \toprule
        \textbf{Dinucleotide} & $P_{1}(R_{i}R_{i+1})$ & $P_{2}(R_{i}R_{i+1})$ & $P_{3}(R_{i}R_{i+1})$ & $P_{4}(R_{i}R_{i+1})$ & $P_{5}(R_{i}R_{i+1})$ & $P_{6}(R_{i}R_{i+1})$\\\midrule
        
        AA & 0.06  &	0.5 	& 0.27 	& 1.59 	& 0.11 	& -0.11 \\\midrule
        AC & 1.50  &	0.50 	& 0.80 	& 0.13 	& 1.29 	& 1.04 \\\midrule
        AG & 0.78  &	0.36 	& 0.09 	& 0.68 	& -0.24 & -0.62 \\\midrule
        AT & 1.07  &	0.22 	& 0.62 	& -1.02 & 2.51 	& 1.17 \\\midrule
        CA & -1.38 & 	-1.36 	& -0.27 & -0.86 & -0.62 & -1.25 \\\midrule
        CC & 0.06  &	1.08 	& 0.09 	& 0.56 	& -0.82 & 0.24 \\\midrule
        CG & -1.66 & 	-1.22 	& -0.44 & -0.82 & -0.29 & -1.39 \\\midrule
        CT & 0.78  &	0.36 	& 0.09 	& 0.68 	& -0.24 & -0.62 \\\midrule
        GA & -0.08 & 	0.5 	& 0.27 	& 0.13 	& -0.39 & 0.71 \\\midrule
        GC & -0.08 & 	0.22 	& 1.33 	& -0.35 & 0.65 	& 1.59 \\\midrule
        GG & 0.06  &	1.08 	& 0.09 	& 0.56 	& -0.82 & 0.24 \\\midrule
        GT & 1.50  &	0.50 	& 0.80 	& 0.13 	& 1.29 	& 1.04 \\\midrule
        TA & -1.23 & 	-2.37 	& -0.44 & -2.24 & -1.51 & -1.39 \\\midrule
        TC & -0.08 & 	0.5 	& 0.27 	& 0.13 	& -0.39 & 0.71 \\\midrule
        TG & -1.38 & 	-1.36 	& -0.27 & -0.86 & -0.62 & -1.25 \\\midrule
        TT & 0.06  &	0.5 	& 0.27 	& 1.59 	& 0.11 	& -0.11 \\

        \bottomrule
    \end{tabular}
\end{table}

Now we can see that the sequence-order effect of a DNA sequence can be, to some extent, reflected through a set of sequence-correlation factors $\theta_{1},...,\theta_{\lambda}$ as clearly defined by Eq.~\ref{eq:PseDNC-thetas} and \ref{eq:PseDNC-5}. Similar to the procedure as described in (7) for converting the amino acid composition to the PseACC, let us augment the DNC of Eq.~\ref{eq:PseDNC-1-1} to the \gls{PseDNC} as given later in the text

\begin{equation}\label{eq:PseDNC-8}
    D = [d_{1} d_{2} ... d_{16} d_{16+1} ... d_{16+\lambda}]^{T}
\end{equation}

where

\begin{equation}\label{eq:PseDNC-9}
    d_{k} = \begin{cases}\frac{f_{k}}{\sum_{i=1}^{16} f_{i} + w\sum_{j=1}^{\lambda}\theta_{j}} & 1 \le k \le 16\\\frac{w\theta_{k-16}}{\sum_{i=1}^{16} f_{i} + w\sum_{j=1}^{\lambda}\theta_{j}} & 17 \le k \le 16 + \lambda\end{cases}
\end{equation}

where $f_{k} (k = 1,2,...,16)$ are the same as those in Eq.~\ref{eq:PseDNC-1-1}, $\theta_{j} (j = 1,2,...,\lambda$ are given by Eq.~\ref{eq:PseDNC-thetas}, $\lambda$ is the number of the total counted ranks (or tiers) of the correlations along a DNA sequence and $w$ is the weight factor. The concrete values for $\lambda$ and $w$ will be discussed further. Thus, instead of a 16-D (dimensional) vector (cf. Eq.~\ref{eq:PseDNC-1-1}), the DNA sequence is now formulated by a $(16 + \lambda) - D$ vector as shown in Eq.~\ref{eq:PseDNC-8}. It is through the additional $\lambda$ correlation factors that not only considerable global sequence-order effects can be incorporated but the DNA sequences with extreme difference in length can also be converted into a set of feature vectors with a same dimension. The latter is an important pre-requisite for formulating the statistical samples because many powerful classification engines, such as Covariant Discriminant (34,35), Support Vector Machine (SVM) (36) and K-Nearest Neighbor (37–39) algorithms, require the input to be a set of digital vectors with a fixed number of components.

%%%%%%%%%%%%%%%%%%%%%%%%%%%%%%%%%%%%%%%%%%%%%%%%%%%%%%%%%%%%%%%%%%%%%%%%%%%%%%%%%%%%%%%%%%%%%%%%%%%%%%%%%%%%%%%%%%%%%%%%%%%%%%%%%%%%%%%%%%%%%%%%%%%%%%%%%%%%%%%%%%%%%%%%%%%%%%%%%%%%%%%%%%%%%%%%%%%%%%%%%%%%%%%%%%%%%%%%%%%%%%%%%%%%%%%%%%%%%%%%%%%%%%%%%%%%%%%%%%%%%%%%%%%%%%

\paragraph{Pseudo K-tupler Composition}

% https://github.com/liufule12/repDNA/blob/master/repDNA/doc/repDNA_manual.pdf

\gls{PseKNC} improved the \gls{PseDNC} approach by incorporating k-tuple nucleotide composition.

Given a DNA sequence D (Eq.~\ref{eq:d-seq}), the feature vector of D is defined:

\begin{equation}\label{eq:PseKNC-feature-vector}
    D = [d_{1}\;d_{2}\;...\;d_{4^{k}}\;d_{4^{k+1}}\;...\;d_{4^{k} + \lambda}]^{T}
\end{equation}

where

\begin{equation}\label{eq:PseKNC-du}
    d_{u} = 
    \begin{cases}
        \frac{f_{u}}{\sum_{i=1}^{4^{k}} f_{i} + w\sum_{j=1}^{\lambda}\theta_{j}} & 1 \le u2 \le 4^{k}
    \\
        \frac{w\theta_{u-4^{k}}}{\sum_{i=1}^{4^{k}} f_{i} + w\sum_{j=1}^{\lambda}\theta_{j}} & 4^{k} \le u \le 4^{k} + \lambda
    
    \end{cases}
\end{equation}

where $\lambda$ is the number of the total counted ranks (or tiers) of the correlations along a DNA sequence; $f_{u}, u\in[1,2,...,4^{k}]$ is the frequency of oligonucleotide that is normalized to $\sum_{i=1}^{4^{k}}f_{i} = 1$; $w$ is a weigth factor; $\theta_{j}$ is given by:

\begin{equation}\label{eq:PseKNC-thetas}
\theta_{j} = \frac{1}{L-j-1} \sum_{i=1}^{L-j-1}\Theta(R_{i}R_{i+1}, R_{i+j}R_{i+j+1}), \;j\in{1,2,...,\lambda;\lambda < L}
\end{equation}

which represents the j-tier structural correlation factor between all the $j^{th}$ most
contiguous dinucleotides. The correlation function $\Theta(R_{i}R_{i+1}, R_{i+j}R_{i+j+1})$ is defined by:

\begin{equation}\label{eq:PseKNC-correlation}
    \Theta(R_{i}R_{i+1}, R_{i+j}R_{i+j+1}) = \frac{1}{\mu}\sum_{v=1}^{\mu}[P_{v} (R_{i}R_{i+1}) - P_{v}(R_{i+j}R_{i+j+1})]^{2}
\end{equation}

where $\mu$ is the number of physicochemical indices, in this study, 6 indices reflecting the local DNA structural properties (Table~\ref{tab:normalized_values}) were employed to generate the \gls{PseKNC} feature vector; $P_{v} (R_{i}R_{i+1})$ represents the numerical value of the v-th physicochemical indices for the dinucleotide $R_{i}R_{i+1}$ at position i and $P_{v} (R_{i+j}R_{i+j+1})$ represents the corresponding value for the dinucleotide $R_{i+j}R_{i+j+1}$ at position i+j.

\subsection{Encoders}

\gls{DNA} sequences consist of continuous sequential letters from a 4-letter alphabet, and, as mentioned in~\ref{sec:dna-dl}, it is first required to transform the string sequence into a numerical value in order to create an input matrix for the model. The encoders that have been implemented are listed below.

\paragraph{One-Hot Encoding}

One-hot encoding is extensively used in deep learning models and is well suited for most models. In addition, the performance of one-hot encoding is very stable across various data sets, however an appropriate model is necessary to get acceptable performance. This approach preserves the positional information of each nucleotide in sequences but disregards high-order relationships between nucleotides~\cite{Zhang2019ModelingNetwork}.

As a result, \gls{A} is encoded to (1, 0, 0, 0), \gls{C} to (0, 1, 0, 0), \gls{G} to (0, 0, 1, 0), and \gls{T} to (0, 0, 0, 1). The final vector's dimension will be $L * 4$ where $L$ is the length of the sequence.

\paragraph{Chemical Encoding}

The four nucleic acids each have unique chemical characteristics~\cite{GolamBari2013DNASequence}. \gls{A} and \gls{G} are purines with two ring structures, whereas \gls{C} and \gls{T} are pyrimidines with one ring structure. \gls{C} and \gls{G} create strong hydrogen bonds while building secondary structures, whereas \gls{A} and \gls{T} form weak hydrogen bonds. In terms of their chemical functionality, \gls{A} and \gls{C} belong to the amino group, while \gls{G} and \gls{T} belong to the keto group. Accordingly, the four nucleic acids may be categorized into three separate categories (Table~\ref{tab:chemical_encoding}).

\begin{table}[ht]
	\caption{Cluster of nucleotides based on chemical properties~\cite{GolamBari2013DNASequence}}
	\label{tab:chemical_encoding}
    \centering
    \begin{tabular}{lll}
    	\toprule
    	\textbf{Chemical property} & \textbf{Class} & \textbf{Nucleotides} \\\midrule
    	
    	\multirow{2}{*}{Ring structure} & Purine & A,G \\
    	& Pyrimidine & C,T\\\midrule
    	
    	\multirow{2}{*}{Hydrogen bond} & Weak & A,T\\
    	& Strong & C,G\\\midrule
    	
    	\multirow{2}{*}{Functional group} & Amino & A,C\\
    	& Keto & G,T\\
        
    	\bottomrule
    \end{tabular}
\end{table}

Three coordinates $(x, y, z)$ were utilized to represent the chemical characteristics of the four nucleotides, and the values 0 and 1 were given to the coordinates in order to incorporate these attributes. If $x$, $y$, and $z$ coordinates respectively represent the ring structure, the hydrogen bond, and the chemical functionality, then each nucleotide may be encoded as ($x_{i}$, $y_{i}$, $z_{i}$), where $x_{i}$ represents the ring structure, $y_{i}$ represents the hydrogen bond, and $z_{i}$ represents the chemical functionality.

As a result, \gls{A} is encoded to (1, 1, 1), \gls{C} to (0, 0, 1), \gls{G} to (1, 0, 0), and \gls{T} to (0, 1, 0). The final vector's dimension will be $L * 3$ where $L$ is the length of the sequence.


\paragraph{K-mer One-Hot Encoding}

As mentioned before, using one-hot encoding on \gls{DNA} sequences solely preserves the positional information of each nucleotide. Recent investigations, however, have shown that including high-order dependencies among nucleotides may enhance the efficacy of DNA models~\cite{Zhang2019ModelingNetwork}. To capture the dependencies, all instances are turned into image-like matrices of high-order relationships using the k-mer encoding method.

For example, in 1-mer one-hot encoding (which is the same as the regular one-hot encoding), each nucleotide is mapped into a vector of size 4 ($A = [1, 0, 0, 0]^{T}$, $C = [0, 1, 0, 0]^{T}$, $G = [0, 0, 1, 0]^{T}$, and $T = [0, 0, 0, 1]^{T}$). The 2-mer encoding is based on the dependencies between two nearby nucleotides, and each dinucleotide is mapped to a 16-dimensional vector ($AA = [1, 0, 0, 0, 0, 0, 0, 0, 0, 0, 0, 0, 0, 0, 0, 0]^{T}$, ..., $TT = [0, 0, 0, 0, 0, 0, 0, 0, 0, 0, 0, 0, 0, 0, 0, 1]^{T})$. The 3-mer encoding is based on the dependencies between three nearby nucleotides, and each dinucleotide is mapped to a 64-dimensional vector ($AAA = [1, 0, 0, 0, ..., 0, 0, 0, 0, 0]^{T}$, ... , $TTT = [0, 0, 0, 0, ..., 0, 0, 0, 0, 0, 1]^{T}$).

As a result, the final vector's dimension will be $(L - k + 1) * 4^{k}$ where $L$ is the length of the sequence.

\section{Classifiers Implementation}

Following the selection of the dataset, the \gls{DNA} sequence records were randomly rearranged and assigned to the training and test/validation sets. A suitable data preprocessing strategy was selected in order to turn the \gls{DNA} sequences, into a numerical representation. This format was necessary to comply with the input requirements of the classification model, which takes only numerical data.

At this stage, there will be either a collection of pre-calculated features, the descriptors, or labeled and encoded training data. The data will be used to train the classification model regardless of the previous choice.

\subsection{Models}

The following sections will provide an insight for each one of the implemented classification models.

\begin{figure}[htbp]
    \centering
    \includegraphics[width=0.7\linewidth]{models_feature_extraction}
    \caption{Models and their feature extraction methods.}
    \label{fig:models_feature_extraction}
\end{figure}

\paragraph{MLP}

The first model is a feedforward artificial neural network called \gls{MLP}. This is the only model that can take the descriptors as input data, since it is only the shallow \gls{ML} model implemented. Figure~\ref{fig:mlp-arch} shows its architecture, which was based in a~\citeauthor{Zhang2020DeepHE:Learning}'s research~\cite{Zhang2020DeepHE:Learning}, one of the cases studies of this thesis.

\begin{figure}[htbp]
    \centering
    \includegraphics[width=0.8\linewidth]{mlp-arch}
    \caption{MLP architecture.}
    \label{fig:mlp-arch}
\end{figure}

\paragraph{CNN}

The \gls{CNN} model, which was also inspired by one of this thesis' case studies, specifically~\citeauthor{Zou2018AGenomics}'s research~\cite{Zou2018AGenomics} and is depicted in Figure~\ref{fig:cnn-arch}. 

\begin{figure}[htbp]
    \centering
    \includegraphics[width=\linewidth]{cnn-arch}
    \caption{CNN architecture.}
    \label{fig:cnn-arch}
\end{figure}

\paragraph{LSTM / BiLSTM}

This \gls{LSTM} model is a simple model regarding the number of layers, but it is possible to pass to the \textit{nn.LSTM} layer a parameter called $num\_layers$ that specifies the number of recurrent layers. Setting $num \_layers = 2$ would result in a stacked \gls{LSTM}, which would consist of two \gls{LSTM}s stacked on top of one another. The second \gls{LSTM} would receive input from the first and compute the final results. Besides, this layer can also take an argument called $bidirectional$, which determines if the \gls{LSTM} is bidirectional or not. This layer can also take a $dropout$ argument that introduces a dropout layer on the outputs of each \gls{LSTM} layer except the last one.

\begin{figure}[htbp]
    \centering
    \includegraphics[width=0.7\linewidth]{lstm-arch}
    \caption{LSTM architecture.}
    \label{fig:lstm-arch}
\end{figure}

\paragraph{GRU / BiGRU} This \gls{GRU} model is identical to the \gls{LSTM}'s, only changing from the \textit{nn.LSTM} layer to \textit{nn.GRU} one. This way it is possible to directly compare these two types of recurrent neural networks.

\begin{figure}[htbp]
    \centering
    \includegraphics[width=0.7\linewidth]{gru-arch}
    \caption{GRU architecture.}
    \label{fig:gru-arch}
\end{figure}

\paragraph{CNN-LSTM / CNN-BiLSTM}

This model is a combination of the \gls{CNN} and \gls{LSTM} models. When using the \gls{CNN} model, the output of the \gls{CNN} is fed into the \gls{LSTM} model. The same previously mentioned properties of \gls{LSTM} are also present (using $num\_layers$ to create a stacked \gls{LSTM} and using $dropout$ to introduce dropout layers on the outputs of each \gls{LSTM} layers except the last one).

\begin{figure}[htbp]
    \centering
    \includegraphics[width=\linewidth]{cnn-lstm-arch}
    \caption{CNN-LSTM architecture.}
    \label{fig:cnn-lstm-arch}
\end{figure}

\paragraph{CNN-GRU / CNN-BiGRU}

This model is a combination of the \gls{CNN} and \gls{GRU} models. When using the \gls{CNN} model, the output of the \gls{CNN} is fed into the \gls{GRU} model. The same previously mentioned properties of \gls{GRU} are also present (using $num\_layers$ to create a stacked \gls{GRU} and using $dropout$ to introduce dropout layers on the outputs of each \gls{GRU} layers except the last one).

\begin{figure}[htbp]
    \centering
    \includegraphics[width=\linewidth]{cnn-gru-arch}
    \caption{CNN-GRU architecture.}
    \label{fig:cnn-gru-arch}
\end{figure}

\subsection{Hyperparameter Tuning}

Additionally, hyperparameter optimization was considered and applied successfully. As mentioned in~\ref{sec:workflow}, the challenge of hyperparameter optimization is selecting the appropriate hyperparameters for a learning algorithm. It may distinguish an ordinary model from a very accurate one. Choosing a different learning rate or modifying the size of a network layer may have a substantial effect on the performance of the model. Ray Tune~\cite{Liaw2018Tune:Training} is the standard tuning tool for hyperparameters and it was used to complete this task. However, before finding the best combination, it is required to define the configuration of the Ray Tune's search space. The implemented one can be found in Table~\ref{tab:search_space}.

\begin{table}[ht]
	\caption{Ray Tune's search space.}
	\label{tab:search_space}
    \centering
    \begin{tabular}{lll}
    	\toprule
    	\textbf{Hyperparameter (x)} & \textbf{Search Space} \\\midrule
    	
    	Hidden Size & $x \in 32, 64, 128, 256$ \\\midrule
        Batch Size & $x \in 8, 16, 32, 64$ \\\midrule
        Learning Rate & $x \in 0.0001, 0.001, 0.01$ \\\midrule
        Dropout & $x \in 0.2, 0.3, 0.4, 0.5$ \\\midrule
        Number of Layers & $x \in 1, 2, 3$ \\
        
    	\bottomrule
    \end{tabular}
\end{table}

Ray Tune will now randomly choose a combination of parameters from these search areas for each trial. It will then train many models in parallel and determine which one has the highest performance. Additionally, the Ray Tune's scheduler \textit{ASHAScheduler} was used, which terminates poorly performing trials early. This is accomplished by choosing a desired metric (loss in this study) and measuring it at the end of each epoch. If the measure keeps worsening, reaching a specified patience value, the trial will end immediately. 

\subsection{Other Parameters}

The process of early stopping is a very useful technique since it prevents wasting time and resources on unnecessary computations. Therefore, this strategy was also implemented in the case when hyperparameter tuning is not being performed, using now the \textit{ReduceLROnPlateau} scheduler from \textit{PyTorch}. This scheduler is also able to reduce the learning rate when the given metric has stopped improving, since models often benefit from a 2-10x reduction in the learning rate when learning has stagnated.
