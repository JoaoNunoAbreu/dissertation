\chapter{Machine and Deep Learning in DNA sequence classification} \label{sec:ml_dl_dna}

\section{Relevant previous work on DNA classification}

\citeauthor{Ma2001DNAStudy}~\cite{Ma2001DNAStudy} presented new biosequence classification algorithms, with an emphasis on detecting E. Coli promoters in \gls{DNA}. They wish to know whether or not an unlabeled \gls{DNA} sequence S is an E. Coli promoter. The -35 and -10 binding sites in an E. Coli promoter sequence are found using an expectation maximization technique. They choose features in each E. Coli promoter sequence based on their information content and express the features using an orthogonal encoding approach based on the located binding sites. The features are then sent into a neural network for promoter detection.

\citeauthor{Chen2002NeuralAnalysis}~\cite{Chen2002NeuralAnalysis} detailed how they used back-propagation, radial basis function networks, self-organizing maps, and committee machines to solve the challenge of gene classification using genomic signatures. The results reveal that these models can achieve average accuracy of 97\% in a two-way classification problem, and average accuracy of more than 83\% in a more challenging four-way classification problem.

\citeauthor{You2009ClassificationCompositions}~\cite{You2009ClassificationCompositions} used an artificial neural network model to classify DNA sequences from numerous microorganisms. To describe the DNA sequences, the dinucleotides compositions approach was utilized, which converts each DNA sequence into a 16-dimensional vector. The leave-one-out approach was used to create and train a back-propagation artificial neural network. The results indicated that the classification accuracy was 84\%, indicating that the model was good overall.

\citeauthor{Nair2010ANNRepresentation}~\cite{Nair2010ANNRepresentation} proposed a unique technique for organism classification based on a combination of \gls{FCGR} and \gls{DNA}. \gls{CGR} displays \gls{DNA} sequences in a unique way and reveals hidden patterns in them. The frequency of sub-sequences contained in the \gls{DNA} sequence is shown by \gls{FCGR}, which is developed from \gls{CGR}. The taxonomic distribution of Eukaryotic species is broken down into eight groups, and \gls{ANN} is used to classify them. Various \gls{ANN} configurations are evaluated, and high accuracy is achieved.

\citeauthor{Rizzo2015AClassification}~\cite{Rizzo2015AClassification} introduced a deep learning neural network based on spectral sequence representation for \gls{DNA} sequence classification. The framework is evaluated on a dataset of 16S genes, and its results are compared to the General Regression Neural Network as well as Naive Bayes, \gls{RF}, and \gls{SVM} classifiers in terms of accuracy and F1 score. When it came to classifying short sequence fragments of 500 bp, the \gls{DL} technique beat all other classifiers, according to the data.

According to~\citeauthor{Zhou2017CNNsite:Features}~\cite{Zhou2017CNNsite:Features}, predicting the \gls{DNA} genome in a big sequence and interacting with \gls{DNA} proteins are difficult operations. They present several ways for the motif feature, which is defined as a subsequence of the \gls{DNA} sequence that is critical for predicting \gls{DNA}-protein interaction. \gls{CNN} captures the motif feature in a sequence and combines it with the evolutionary binding feature in a sequence. To analyze the effectiveness of \gls{CNN}, the authors employed two datasets: PDNA-62 and PDNA-224.

\citeauthor{Nguyen2016DNANetwork}~\cite{Nguyen2016DNANetwork} developed a new method for classifying \gls{DNA} sequences using a \gls{CNN} while treating them as text input. Because the authors employed one-hot vectors to represent sequences as input to the model, the important position information of each nucleotide in sequences is preserved. The authors investigated the suggested model using 12 \gls{DNA} sequence datasets and found substantial improvements in all of them. This finding suggests that a \gls{CNN} can be used to handle additional sequence challenges in bioinformatics.

\citeauthor{LoBosco2017DeepClassification}~\cite{LoBosco2017DeepClassification} offer two distinct \gls{DL} architectures (\gls{CNN} and \gls{RNN}) for \gls{DNA} sequence classification. For five separate classification tasks, they compare their results using a public data set of \gls{DNA} sequences. Two \gls{DL} architectures were examined in this work for automated classification of bacteria species with no sequence preprocessing procedures. In comparison to a traditional \gls{CNN}, the authors have presented a \gls{LSTM} that utilizes the nucleotide locations in a sequence. The \gls{CNN}s outperform the \gls{LSTM} in the four easiest classification tasks, but their performance deteriorates in the last, when the \gls{LSTM} performs better.

\citeauthor{Solis-Reyes2018DNADatasets}~\cite{Solis-Reyes2018DNADatasets} presented a sequence classification approach based on k-mer proportions of \gls{DNA} sequences that is open-source, supervised, alignment-free, and extremely generic. The approach was implemented in Kameris, a completely freestanding general-purpose software package that is freely accessible under an open-source license with a permissive license. The program has significant benefits over rival software in terms of data security and privacy, openness, and repeatability. The authors investigated its accuracy and performance across a range of classification tasks, including viral subtyping, taxonomy categorization, and human haplogroup assignment. On complete mitochondrial, nuclear, plastid, plasmid, and viral genomes, as well as randomly chosen eukaryote genomes and transcriptomes, the authors demonstrated the effectiveness of the proposed technique.

\citeauthor{Abd-Alhalem2020BacterialLayers}~\cite{Abd-Alhalem2020BacterialLayers} proposed, based on a custom layer, a new technique for classification of bacterial \gls{DNA} sequences. The \gls{FCGR} of \gls{DNA} is employed with a \gls{CNN}. With a proper choice of the frequency k-lengthen words occurrence in \gls{DNA} sequences, the \gls{FCGR} is used as a sequence representation technique. The \gls{DNA} sequence is mapped using \gls{FCGR}, which generates a gene sequence image. Both local and global patterns may be seen in this sequence. For image classification, a pre-trained \gls{CNN} is used. 

\citeauthor{Chen2017AClassification}~\cite{Chen2017AClassification} explain the distinction between cell-free DNA and normal DNA. Cell-free DNA (cfDNA) has a specific fragmentation pattern that is nonrandom and differs from regular DNA in structural and sequential features. The authors employed a variety of classification models to categorize normal and cfDNA, including \gls{KNN}, \gls{SVM}, and \gls{RF}, to achieve an accuracy of above 98\%. In comparison to all other algorithms, the authors determined that the \gls{RF} had the best accuracy. The data utilized and validated are cfDNA and gDNA (genome) collected from the same blood samples, with cfDNA extracted from the plasma of blood cells.

\citeauthor{Helaly2019ConvolutionalStudy}~\cite{Helaly2019ConvolutionalStudy} analyze three of the most current \gls{DL} efforts for taxonomy classification using the 16S rRNA barcode dataset. Three distinct \gls{CNN} architectures are examined, as well as three different feature representations: k-mer spectral representation, \gls{FCGR}, and character-level integer encoding. The most fine-grained classification challenge showed that representations that hold positional information about the nucleotides in a sequence perform substantially better, with accuracies reaching 91.6\%.

\citeauthor{Gunasekaran2021AnalysisModels}~\cite{Gunasekaran2021AnalysisModels} used \gls{CNN}, \gls{CNN}-\gls{LSTM}, and \gls{CNN}-Bidirectional \gls{LSTM} architectures with Label and k-mer encoding for \gls{DNA} sequence classification. Different classification metrics were used to evaluate the models. According to the findings of the experiments, the \gls{CNN} and \gls{CNN}-Bidirectional \gls{LSTM} with k-mer encoding have good accuracy, with 93.16\% and 93.13\% on testing data, respectively.

\citeauthor{Soliman2021AnClassification}~\cite{Soliman2021AnClassification} proposed a modified \gls{CNN} to create an integrated model for \gls{DNA} classification. The \gls{CNN}s are investigated by altering hyperparameters such as the learning rate, minibatch size, and epoch count. On 16S rRNA bacterial sequences, \gls{CNN}s are trained and their performance is evaluated. With a learning rate of 0.0001, a size of minibatch equal to 64, and a number of epochs equal to 20, simulation results show that using a \gls{CNN} based on wavelet subsampling gives the optimal trade-off between processing time and accuracy.

\citeauthor{Lugo2021AIdentification}~\cite{Lugo2021AIdentification} presented a sequential \gls{DL} approach for bacterium identification. To derive an identification model for whole-genome bacterium sequences, the proposed neural network takes advantage of the massive volumes of data supplied by Next-Generation Sequencing. The bidirectional \gls{RNN} (BI-\gls{GRU}) outperformed other classification algorithms (Naive Bayes, Multilayer Perceptron) after verifying the identification model. In a low-dimensional space, a distributed representation was proven as the appropriate encoding for bacterial genetic information. The distributed representation is given context by combining two or more k-mer lengths. Context makes use of positional data, which is critical in biological sequences. The new context proved advantageous for the identification model efficiency in Natural Language Processing applications. 

% das 2 abordagens
% TABELA

\section{Tools for building DNA sequence classification algorithms}

\section{Notas (TEMPORÁRIO)}

Understanding the connections between protein structure and function is one of biology's main goals. The basic amino acid sequences provide particularly helpful structural information for understanding the structure-function paradigm. The idea that sequences with similar structures have comparable functions is used to classify DNA sequences. Sequence alignment techniques such as BLAST~\cite{Altschul1990BasicTool} and FASTA~\cite{Pearson1988ImprovedComparison} have historically been used to determine sequence similarity. This decision is based on two primary assumptions: (1) the functional components have similar sequence properties, and (2) the functional elements' relative order is preserved across sequences. These assumptions are applicable in a wide range of situations, but they are not universal~\cite{LoBosco2017DeepClassification}.

Regardless, despite recent advancements, the major issue that severely restricts the use of alignment techniques remains their computational time complexity. As a result, alignment-free approaches~\cite{Vinga2003Alignment-freeReview,Pinello2014ApplicationsEpigenomics} that have recently been developed have emerged as a viable strategy to investigating the regulatory genome. Feature extraction, such as spectral representation of DNA sequences~\cite{LoBosco2014ASequences,LoBosco2015AlignmentClassification}, is undoubtedly a part of these approaches. Feature representations, which are often constructed by specialists, are critical to the effectiveness of traditional machine learning algorithms. Following that, it is vital to choose which features are more suited for dealing with the particular problem, which is still a very important and challenging stage today~\cite{LoBosco2017DeepClassification}.

Limitations and suggestions for \gls{ML}~\cite{Abd-Alhalem2021DNASurvey}.

\begin{itemize}
    \item \textbf{Data Representation}: Quantifying characteristics of DNA sequences presents various challenges. Nobody knows which format for encoding numeric values in these nucleotides is the best. When applying learning machines to biological investigations, however, we cannot escape employing quantitative representations of these biological units.
    
    \item \textbf{Reducing computational requirement}: Deep learning models are often complicated and have a large number of parameters to train; obtaining well-trained models, as well as using the models productively, is generally computationally and memory costly. These limitations severely limit the use of deep learning in machines with low processing capability, particularly in the data-intensive fields of bioinformatics and healthcare. Several methods have been proposed to compress deep learning models, which can significantly reduce the computational requirements of those models from the start, such as parameter pruning, which significantly reduces redundant parameters that do not contribute to the model's performance, and the well-known Deep Compresion. We may also conserve parameters by using compact convolutional filters.
    
    \item \textbf{Hybrid methods}: refers to combining several deep learning models, such as CNN as a feature extraction step with an RNN classifier to provide more knowledge, to get superior results. This might bring fresh perspectives and fine insights into the nature of the genome. 
    
    \item Trying to extract other features from DNA sequences before feeding them to the algorithms. As a result, overall evaluations may become more accurate.
\end{itemize}