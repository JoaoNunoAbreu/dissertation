\chapter{Machine and Deep Learning in DNA sequence classification} \label{sec:ml_dl_dna}

\section{Relevant previous work on DNA classification}

\textbf{ML}

analyse and classify proteins based on sequence-derived properties focusing on the protein feature extraction
\begin{itemize}
    \item Propy
    \item PyDPI
    \item PyBioMed
    \item pfeature
\end{itemize}

able to calculate features and to perform ML
\begin{itemize}
    \item Bio-seq analysis
    \item ModlAMP
    \item iFeature
    \item iLearn
\end{itemize}

\textbf{DL}

\begin{itemize}
    \item i dont know man 
\end{itemize}

% das 2 abordagens
% TABELA

\section{Tools for building DNA sequence classification algorithms}

\section{Notas (TEMPORÁRIO)}

Understanding the connections between protein structure and function is one of biology's main goals. The basic amino acid sequences provide particularly helpful structural information for understanding the structure-function paradigm. The idea that sequences with similar structures have comparable functions is used to classify DNA sequences, which is particularly beneficial for this purpose. Sequence alignment techniques such as BLAST~\cite{Altschul1990BasicTool} and FASTA~\cite{Pearson1988ImprovedComparison} have historically been used to determine sequence similarity. This decision is based on two primary assumptions: (1) the functional components have similar sequence properties, and (2) the functional elements' relative order is preserved across sequences. These assumptions are applicable in a wide range of situations, but they are not universal~\cite{LoBosco2017DeepClassification}.

Regardless, despite recent advancements, the major issue that severely restricts the use of alignment techniques remains their computational time complexity. As a result, alignment-free approaches~\cite{Vinga2003Alignment-freeReview,Pinello2014ApplicationsEpigenomics} that have recently been developed have emerged as a viable strategy to investigating the regulatory genome. Feature extraction, such as spectral representation of DNA sequences~\cite{LoBosco2014ASequences,LoBosco2015AlignmentClassification}, is undoubtedly a part of these approaches. Feature representations, which are often constructed by specialists, are critical to the effectiveness of traditional machine learning algorithms. Following that, it is vital to choose which features are more suited for dealing with the particular problem, which is still a very important and challenging stage today~\cite{LoBosco2017DeepClassification}.

DL - \textbf{low level cost of parallel computing architectures.}