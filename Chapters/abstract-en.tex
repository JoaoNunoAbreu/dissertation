{\large \textbf{\thetitle}}\\[1ex]
\noindent Deoxyribonucleic acid (DNA) is a biological macromolecule whose primary function is to store information. DNA sequences hold an individual's genetic information and contain almost all the information about an individual's health and physical well-being. Because of breakthroughs in sequencing technology, the number of these sequences is now growing at an exponential rate. Machine learning has been widely employed as it is a strong tool for processing huge amounts of data by learning on its own without explicit programming. Using machine learning, it is now possible to speed up and automatically classify DNA sequences into existing categories with the objective of learning their functions. For example, viral identification and classification are crucial in order to prevent an epidemic like \emph{COVID-19}.

However, building a machine learning classifier of biological sequences is a tough challenge due to the lack of numerical properties in the sequence that the model requires. Therefore, it is still necessary to apply some pre-processing techniques so that the sequences are properly represented for the model. These techniques include feature extraction and feature selection, and they are the most difficult components because sequences lack explicit features. Deep learning models have recently been developed that not only extract features from input automatically but also improve the prediction and classification of DNA sequences.

The main goal of this project is to create a tool that can automatically classify DNA sequences using machine and deep learning models and algorithms, followed by its integration into \textit{ProPythia}. Automated machine learning classifiers will also be developed to integrate in \textit{OmniumAI} software platforms. Transcription factor annotation and essential gene determination will be used as case studies for the platform validation. With this study, it is intended to encourage the use of such technologies to develop new tools that can manage vast volumes of biological data, thus boosting DNA prediction understanding.

\begin{keywords}
DNA, DNA Sequence Classification, Machine Learning, Deep Learning
\end{keywords} 
