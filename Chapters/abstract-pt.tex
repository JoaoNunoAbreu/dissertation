{\large \textbf{Desenvolvimento de classificadores de sequências de ADN baseado em deep learning}}\\[1ex]
\noindent O ácido desoxirribonucleico (ADN) é uma macromolécula biológica cuja principal função é armazenar informações. As sequências de ADN contêm as informações genéticas de um indivíduo e as suas informações sobre a saúde e o bem-estar físico. Devido aos avanços na tecnologia de sequenciamento, o número dessas sequências está a crescer a uma taxa exponencial. Machine learning tem sido bastante utilizado, pois é uma ferramenta capaz de processar grandes quantidades de dados aprendendo por conta própria sem programação explícita. Desta maneira, é possível acelerar e classificar automaticamente as sequências de ADN em categorias existentes com o objetivo de aprender as suas funções. Por exemplo, a identificação e classificação de vírus são cruciais para prevenir uma epidemia como \emph{COVID-19}.

No entanto, construir um classificador de machine learning de sequências biológicas é um grande desafio devido à falta de propriedades numéricas na sequência que o modelo exige. É necessário aplicar algumas técnicas de pré-processamento para que as sequências sejam devidamente representadas para o modelo. Essas técnicas incluem extração e seleção de características, e são os componentes mais difíceis porque as sequências carecem de características explícitas. Modelos de deep learning foram desenvolvidos recentemente que não só extraem características dos dados automaticamente, como também melhoram a previsão e classificação de sequências de ADN.

O principal objetivo deste projeto é criar uma ferramenta capaz de classificar automaticamente sequências de ADN usando modelos e algoritmos de machine e deep learning, seguido da sua integração no \textit{ProPythia}. Classificadores automáticos de machine learning também serão desenvolvidos para integração em plataformas de software \textit{OmniumAI}. A determinação do fator de transcrição e de genes essenciais serão utilizados como casos de estudo para validação da plataforma. Com este estudo, pretende-se incentivar o uso de tais tecnologias para desenvolver novas ferramentas que consigam lidar com grandes volumes de dados, permitindo avanços na área de previsão de ADN.

% Palavras-chave do resumo em Português
\begin{keywords}
ADN, Classificação de sequências de ADN, Machine Learning, Deep Learning
\end{keywords}
% to add an extra black line
