\section{Technologies used}

This section will go through the main technologies employed in this project that were very crucial in achieving the final result.

\subsection{The Software}

\textit{Python} is a high-level interpreted general-purpose programming language that supports vast and extensive external libraries that are constantly evolving. \textit{PyCharm Code Editor} was used to combine Python code development while also improving usability and user experience. Additionally, the \textit{Anaconda} software application was employed to facilitate package management and distribution.

\subsection{The Hardware}

\subsection{The Libraries}

PyTorch and Tensorflow are the most popular and powerful \gls{ML} frameworks, and one had to be chosen to be used in this research for the implementation of the models. 

%%%%%%%%%%%%%%%%%%%%%%%%%%%%%%%%%%%%%%%%%%%%%%%%%%%%%%%%%%%%%%%%%%%%%%%%%%%%%%%%%%%%%%%%
%% VER SE VALE A PENA DIZER O PORQUE DE TER SIDO ESCOLHIDO PYTORCH EM VEZ DE KERAS / TENSFORFLOW
%%%%%%%%%%%%%%%%%%%%%%%%%%%%%%%%%%%%%%%%%%%%%%%%%%%%%%%%%%%%%%%%%%%%%%%%%%%%%%%%%%%%%%%%%%%%%%%%%%%%%%%%%%%%%%%%%%%%%%%%

\textit{PyTorch} is a free and open-source \gls{ML}/\gls{DL} library that was used in this research to create the all the models. 

\textit{Numpy} is a key Python library for scientific computing that supports massive, multidimensional arrays, complex matrix arithmetic, and much more. This package was highly helpful for data pre-processing as well as matrix arithmetic calculation in the classication model.

Last but not least, \textit{Matplotlib} is a \textit{Python} package used to plot 2D graphs and other high-quality figures. The latter were based on data obtained from the classication model.