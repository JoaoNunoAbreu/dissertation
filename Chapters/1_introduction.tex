%!TEX root = ../template.tex
%%%%%%%%%%%%%%%%%%%%%%%%%%%%%%%%%%%%%%%%%%%%%%%%%%%%%%%%%%%%%%%%%%%
%% chapter1.tex
%% NOVA thesis document file
%%
%% Chapter with introduciton
%%%%%%%%%%%%%%%%%%%%%%%%%%%%%%%%%%%%%%%%%%%%%%%%%%%%%%%%%%%%%%%%%%%
\newcommand{\novathesis}{\emph{novathesis}}
\newcommand{\novathesisclass}{\texttt{novathesis.cls}}


\chapter{Introduction}
\label{cha:introduction}

\section{Context and Motivation} \label{sec:context_and_motivation}

Biomedical data has grown at an exponential rate in recent years, requiring the use of a variety of machine learning approaches to handle new issues in biology and clinical research. Machine learning methods are often integrated with bioinformatics methodologies, as well as curated databases and biological networks, to improve training and validation, find the most interpretable features, and enable feature and model research~\cite{Auslander2021IncorporatingFrameworks}.

In organisms, DNA is a biomacromolecule. It holds life's genetic information and controls biological growth as well as the proper functioning of life's functions. Machine learning is now frequently utilized in sequence data analysis, and it has a wide range of applications in terms of enhancing data processing capacities and providing useful biological data~\cite{Yang2020ReviewDNA}.

The assignment of a function to a sequence representing a part of a DNA molecule is a core problem in Bioinformatics, extremely important for biomedical research. Current solutions involve the use of homologies, inferred by sequence similarity, i.e., classifying new sequences based on known functions in sequences with a high degree of similarity. Deep neural networks are an alternative that can automatically learn and comprehend informative sequence representations to get a better understanding of the regulatory code that governs gene expression~\cite{Zrimec2021LearningExpression}.

There is already a platform, developed within the Biosystems group at CEB/ U. Minho, devoted to the classification of peptides/proteins sequences using machine learning and deep learning called \textit{ProPythia}~\cite{Sequeira2020ProPythia:Learning}. One of the objectives of this thesis will be the integration of a tool to support DNA sequence classifiers on the mentioned platform.

\section{Research Objectives} \label{sec:research_objectives}

The main aim of this work is to develop an automatic classification system for DNA sequences using machine and deep learning algorithms, expecting performance gains in terms of response time to annotate large numbers of sequences (e.g., complete genomes), as well as the accuracy of the results obtained. 

In detail, the work will address the following scientific/technological objectives:

\begin{itemize}
    \item Review relevant literature and existing tools regarding deep learning methods and their applications in sequence classification.
    \item Develop and compare data pre-processing techniques and understand the impact of different methods on the classifying performance of machine and deep learning models.
    \item Develop a tool to support machine and deep learning models for DNA sequence classification to integrate in \textit{ProPythia}.
    \item Develop automated machine learning (AutoML) classifiers for DNA sequences, which will be integrated into \textit{OmniumAI} software platforms.
    \item Validate the developed tool and platform with case studies in the areas of biotechnology and health, e.g., transcription factor annotation and essential genes determination~\cite{Zhang2020DeepHE:Learning,Quang2016DanQ:Sequences,Novakovsky2021BiologicallyPrediction}.
    \item Write the master thesis.
\end{itemize} 


\section{Document Structure} \label{sec:document_structure}

This thesis is divided into seven chapters, each of which is briefly described as follows:

\begin{itemize}
    \item Chapter 1: Overview of the work's subject, as well as the motivation and key objectives.
    \item Chapter 2: Theoretical concepts of machine and deep learning, as well as their objectives, algorithms, workflows, and architectures. 
    \item Chapter 3: Introduction of DNA and DNA sequence classification topics. Applications of machine and deep learning in DNA sequence classification, as well as relevant previous work.
    \item Chapter 4: Decisions and methods for implementing the proposed work, including details on the feature extraction and classification models used. 
    \item Chapter 5: Integration of the developed tool into \textit{ProPythia} and \textit{OmniumAI} software platforms.
    \item Chapter 6: A summary of the case study datasets, how they were acquired, and the tool's effectiveness on them.
    \item Chapter 7: Brief summary of the dissertation, followed by a discussion of the goals, which are presented in this chapter, and the results from the previous chapter. Also provided at the end are some suggestions for future work.
\end{itemize}