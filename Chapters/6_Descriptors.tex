\chapter{Descriptors (falta quillbot'ar este cap todo)}
\label{cha:descriptors}

In this work, we first examined the state-of-the-art in computationally predicting ... and then comprehensively assessed the predictive power of eight traditional machine learning methods and 17 feature types often used in prior research.

Based on the sequence and physicochemical characteristics, we created and evaluated a total of 17 different kinds of features.

\section{Length}

Length descriptor is a simple descriptor that calculates the length of a sequence.

\section{GC Content}

\gls{GC} content feature enconding represents the quantity of guanine and cytosine nucleotides in a sequence. It can be calculated as follows:

\begin{equation}\label{eq:gc_content}
    x = \frac{N_{(C)} + N_{(G)}}{N}
\end{equation}


where $N_{(C)}$ donates the number of the nucleotide C in the sequence, $N_{(G)}$ the number of the nucleotide G and $N$ the length of the sequence.

\section{AT Content}

\gls{AT} content feature enconding represents the quantity of adenine and thymine nucleotides in a sequence. It can be calculated as follows:


\begin{equation}\label{eq:at_content}
    x = \frac{N_{(A)} + N_{(T)}}{N}
\end{equation}

where $N_{(A)}$ donates the number of the nucleotide A in the sequence, $N_{(T)}$ the number of the nucleotide T and $N$ the length of the sequence.

\section{Nucleic Acid Composition}

As one of the commonly used methods to represent \gls{DNA} sequences, the \gls{NAC} encoding [25, 26] reflects the nucleotides frequencies of the sequence. The frequencies of all four natural nucleotides ('A', 'C', 'G' and 'T') can be calculated as:

\begin{equation}\label{eq:NAC}
    f(i) = \frac{N_{(i)}}{N}, i \in \left\{A,C,T,G\right\}
\end{equation}

where $N_{(i)}$ donates the number of nucleotide type and $N$ represents the length of a \gls{DNA} sequence.

\section{Di-Nucleotide Composition}

\gls{DNC} feature encoding [27, 28] represents the composition of continuous dinucleotide pairs in a \gls{DNA} sequence. There are 16 descriptors in \gls{DNC} feature encoding, which can be defined as:

\begin{equation}\label{eq:DNC}
    D(i,j) = \frac{N_{(ij)}}{N-1}, i,j \in \left\{A,C,T,G\right\}
\end{equation}


where $N_{(ij)}$ donates the number of dinucleotide represented by nucleotide types $i$ and $j$.

\section{Tri-Nucleotide Composition}

\gls{TNC} feature encoding [29, 30] represents the composition of the composition of continuous trinucleotide pairs in a \gls{DNA} sequence. There are 64 descriptors in \gls{TNC} feature encoding ('AAA', 'AAC', 'AAG', 'AAT', ..., 'TTT'), which can be defined as:

\begin{equation}\label{eq:TNC}
    D(i,j,k) = \frac{N_{(ijk)}}{N-2}, i,j,k \in \left\{A,C,T,G\right\}
\end{equation}

where $N_{(ijk)}$ donates the number of trinucleotides represented by nucleotide types $i$, $j$ and $k$.


\section{Composition of K-spaced nucleic acid pairs}

\gls{CKSNAP} feature encoding [21, 22] represents the composition of nucleotide pairs that are K-steps away from each other in a segment. Specifically, we calculated the frequency of a nucleotide pair with the two nucleotides at positions $i$ and $i + K + 1$, respectively, where $i$ = 1, ..., ($l$ − $K$ − 1) and l being the length of the sequence. For example, given the sequence $ACGTACGT$ and $K$ = 2, the nucleotide AT will occur twice in the sequence, where A and T occur at positions 1 and 4 and also at positions 5 and 8. 

It is important to note that there are only a total of 16 possible nucleotide pairs regardless of the value of $K$. This coding system reflects the short-range interactions of nucleic acids within a \gls{DNA} sequence segment.

\section{K-mer}

The K-mer encoding [34, 35] calculates the occurrence frequencies of k neighboring nucleotide in a \gls{DNA} sequence, which was commonly used in the field of enhancer identification and regulatory sequence prediction (2). The K-mer (k = 4) descriptor can be defined as:

\begin{equation}\label{eq:K-mer}
    K(i) = \frac{N_{(i)}}{N}, i \in \left\{AAAA, AAAC, AAAG,...,TTTT\right\}
\end{equation}

where $N_{i}$ donates the number of types $i$ descriptor of K-mer and $N$ representes the length of the DNA sequence.

The implemented K-mer descriptor also includes the \gls{RCKmer}. The \gls{RCKmer} encoding [36] is a variant of K-mer descriptor, which calculates the occurrence frequencies of reverse compliment k neighboring nucleotide in the \gls{DNA} sequence. For example, there are 16 types of 2-mers in a \gls{DNA} sequence. Among them, 'TT' is reverse compliment with 'AA'. Thus, there are only 10 types of 2-mers in the \gls{RCKmer} approach (i.e. 'AA', 'AC', 'AG', 'AT', 'CA', 'CC', 'CG', 'GA', 'GC' and 'TA') by removing the reverse complimentary K-mers.

\section{Accumulated Nucleotide Frequency}

The nucleotide density and distribution of each nucleotide in a \gls{DNA} segment are represented by the \gls{ANF} feature encoding scheme [17]. The formula below describes how to calculate the \gls{ANF} of a \gls{DNA} segment of length L.

\begin{equation}\label{eq:K-mer}
    d_{l} = \frac{1}{l}\sum_{j=1}^{l}f(n_{j}), f(n_{j}) = \begin{cases}1 & n_{j} = q\\0 & other\end{cases}, l = 1,...,L
\end{equation}

where $n_{j}$ represents the nucleotide at the j-th position and $q \in (A,C,T,G)$. Taking the sequence of 'TGCTACGC' as an example, when $l$ = 3, the nucleotide at the $l$-th position is C and the density of this position is calculated as $d_{3} = \frac{1}{3}\sum_{j=1}^{3}f(n_{j}) = \frac{1}{3} [f(C) + f(A) + f(C)] = \frac{1}{3} [1+0+1] = 0.667$. The density of all L positions can be similarly calculated. However, if we calculate the ANF for each position, we would get L values, which is not a fixed length vector. To avoid this, we only calculated the ANF for 3 positions, which were at 25, 50 and 75\% of the length of the sequence.

\section{DAC}
\section{DCC}
\section{DACC}
\section{TAC}
\section{TCC}
\section{TACC}
\section{PseDNC}
\section{PseKNC}